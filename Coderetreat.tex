%\def\ishandout{}
\ifdefined\ishandout
\PassOptionsToClass{handout}{beamer}
\fi
\documentclass[12pt, xcolor=table, dvipsnames]{beamer}
%\usepackage[utf8]{inputenc}

\usepackage[english]{babel}
\usepackage{fontspec} % loads: fixltx2e, etex, xunicode, fontspec.
\setmainfont{Arial}

\setlength{\parskip}{1ex}

\usepackage{fixme}

\usepackage{graphicx}
%\DeclareGraphicsExtensions{.pdf,.png,.jpg}
\graphicspath{{img/}}


%%%%%% Overpic + Textpos
\usepackage[permil]{overpic}

\usepackage[absolute,overlay,showboxes]{textpos}
\setlength\unitlength{1mm}
\setlength{\TPHorizModule}{1mm}
\setlength{\TPVertModule}{\TPHorizModule}
\textblockorigin{0mm}{0mm} % start everything near the top-left corner
\TPshowboxesfalse

%%%%%%%%%
\usepackage{color}

\usepackage{listings}
\definecolor{javared}{rgb}{0.6,0,0} % for strings
\definecolor{javagreen}{rgb}{0.25,0.5,0.35} % comments
\definecolor{javapurple}{rgb}{0.5,0,0.35} % keywords
\definecolor{javadocblue}{rgb}{0.25,0.35,0.75} % javadoc

\lstset{ %
    language=C++,                % choose the language of the code
    basicstyle=\tiny\ttfamily,       % the size of the fonts that are used for the code
    %numbers=left,                   % where to put the line-numbers
    %numberstyle=\footnotesize,      % the size of the fonts that are used for the line-numbers
    %stepnumber=1,                   % the step between two line-numbers. If it is 1 each line will be numbered
    %numbersep=5pt,                  % how far the line-numbers are from the code
    backgroundcolor=\color{white},  % choose the background color. You must add \usepackage{color}
    showspaces=false,               % show spaces adding particular underscores
    showstringspaces=false,         % underline spaces within strings
    showtabs=false,                 % show tabs within strings adding particular underscores
    frame=single,           % adds a frame around the code
    tabsize=2,          % sets default tabsize to 2 spaces
    %captionpos=b,           % sets the caption-position to bottom
    %breaklines=true,        % sets automatic line breaking
    %breakatwhitespace=false,    % sets if automatic breaks should only happen at whitespace
    %escapeinside={\%*}{*)}          % if you want to add a comment within your code
    keywordstyle=\color{javapurple}\bfseries,
    stringstyle=\color{javared},
    commentstyle=\color{javagreen},
    morecomment=[s][\color{javadocblue}]{/**}{*/},
}

\usepackage{multicol}

%%%%% Beamer %%%%%
%\usetheme{UniversiteitGent}
\usetheme{Madrid}
\usecolortheme{seagull}
\usepackage[german=guillemets]{csquotes}

\newlength{\templenght}
\newcommand{\textttb}[1]{\texttt{\bf#1}}

\usepackage{etoolbox}
\makeatletter
\patchcmd{\beamer@sectionintoc}{\vskip1.5em}{\vskip0.5em}{}{}
\makeatother

\AtBeginSection % Do nothing for \section*
{
	\begin{frame}[squeeze]
	\frametitle{Outline}
	\small
	\tableofcontents[currentsection]
\end{frame}
}

\newcommand{\red}[1]{{\color{red}#1}}
\newcommand{\green}[1]{{\color{ForestGreen}#1}}


\begin{document}
\author[Lechner]{Lechner} %Teiniker/Schindler
\title{Coderetreat }
\subtitle{Graz 2019}
%\logo{}
\institute[FH JOANNEUM]{FH JOANNEUM\\
	Internettechnik\\
	\url{http://www.fh-joanneum.at}}
%
%\subject{}
%\setbeamercovered{transparent}
\setbeamertemplate{navigation symbols}{}
\frame[plain]{\maketitle}

%%%%%%%%%%%%%%%%%%%%%%%% Frame %%%%%%%%%%%%%%%%%%%%%%%%%%
\begin{frame}[squeeze]{Outline}
\small
\tableofcontents
\end{frame}

%CodeRetreat2019.png  ITCS_LOGO_small.jpg


> 08:30 Eintreffen und Frühstück
> 09:00 Begrüßung, Erklärung des Ablaufs, drei Aufgaben am Vormittag
> 12:45 Mittagspause - für das Mittagessen ist gesorgt!
> 13:45 Zwei Aufgaben am Nachmittag
> 15:45 Wrap Up :)


%%%%%%%%%%%%%%%%%%%%%%%%%%%%%%%%%% Frame %%%%%%%%%%%%%%%%%%%%%%%%%%%%%%%%%%%%%%%%%%
\begin{frame}[plain]
\begin{center}
    \textit{\color{blue}{\large
            Writing clean code is what you must do in order to call yourself a professional.\\
            There is no reasonable excuse for doing anything less than your best.}}
\end{center}

\flushright--Robert C. Martin
\end{frame}

\section{Introduction}
\subsection{What is a Code Retreat}
%%%%%%%%%%%%%%%%%%%%%%%%%%%%%%%%%% Frame %%%%%%%%%%%%%%%%%%%%%%%%%%%%%%%%%%%%%%%%%%
\begin{frame}[squeeze]{\subsecname{}}

\end{frame}



\subsection{Work as you should}
\begin{frame}[squeeze]{\subsecname{}}
	Gap between: 
	How we work - how we should work 
	-> How much we suck

	No pressure - try to suck less :)	
	\begin{itemize}
		\item Pair Programming (PP)
		\item Test Driven Development (TDD)
		\item 4 Rules of simple Design	+ Clean Code 	
	\end{itemize}
\end{frame}

\subsection{4 Rules of simple design}
\begin{frame}{\subsecname{} (by Kent Beck)}
	Simple design:\pause
	\begin{itemize}[<+->]
		\item \textbf{\color{red} Passes the tests}
		\item Reveals all intention (Naming!)
		\item Contains no duplication (DRY)
		\item Fewest elements (KISS)
	\end{itemize}
\end{frame}


\section{Sessions}

> 08:30 Eintreffen und Frühstück
> 09:00 Begrüßung, Erklärung des Ablaufs, drei Aufgaben am Vormittag
> 12:45 Mittagspause - für das Mittagessen ist gesorgt!
> 13:45 Zwei Aufgaben am Nachmittag
> 15:45 Wrap Up :)


\subsection{Session 1}
\begin{frame}[squeeze]{\subsecname{}}
	Get started:
	\begin{itemize}
		\item Understand the problem 
		\item Get used to the setting (PP, TDD, Clean Code)
		\item No further constraints
	\end{itemize}
\end{frame}


\subsection{Session 2}
\begin{frame}[squeeze]{\subsecname{}}
	Pay attention to the working style
	\begin{itemize}
		\item Fokus on PP, TDD, Clean Code
		\item Constraints: Ping-Pong Pairing
	\end{itemize}
\end{frame}

\subsection{Session 3}
\begin{frame}[squeeze]{\subsecname{}}
\end{frame}

\subsection{Lunchbreak}
\begin{frame}[squeeze]{\subsecname{}}
\end{frame}

\subsection{Session 4}
\begin{frame}[squeeze]{\subsecname{}}
\end{frame}

\subsection{Session 5}

%https://www.coderetreat.org/pages/hosting/hosts/

%Ping Pong
%Silent pairing
%Legacy code (Switch Laptop)
%5min Limit (Git)
%Tell dont ask
%No Primitives
%Classes as Verbs instead of nouns
%Immutables only, please (vielleicht mit etwas anderem kombinieren)
%No conditional statements
%If you feel people are not refactoring enough – choose one of the “Quality Constraints”. My favorite is – “No methods bigger than 5 lines”. Closing brackets do not count in those 5. For scripting languages like Ruby, Javascript and Python constraint is 3 lines.
%People are using a lot of primitives? “No Naked Primitives”
%To make them try new stuff or remember something forgotten try “No Loops” or “No If Statements”.

%https://docs.google.com/document/d/1W32Qh4zU2Y-QRdMkbF5zbqUGJM7wMmcM3DoukJLSLq8/edit

\section{Closing}
%%%%%%%%%%%%%%%% Frame %%%%%%%%%%%%%
\begin{frame}{\secname{} }
    \begin{itemize}[<+->]
        \item xxx
    \end{itemize}
\end{frame}

\section{Reference}
%%%%%%%%%%%%%%%%%%%%%%%%%%%
\begin{frame}{\secname{}}
    \begin{itemize}[<*>]
        \item Clean Code: A Handbook of Agile Software Craftsmanship
        \begin{itemize}[<*>] 
        \item Robert C. Martin
        \item 2008 Prentice Hall
        \end{itemize} 
     \end{itemize}
\end{frame}





\end{document}