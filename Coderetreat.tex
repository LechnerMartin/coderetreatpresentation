%\def\ishandout{}
\ifdefined\ishandout
\PassOptionsToClass{handout}{beamer}
\fi
\documentclass[12pt, xcolor=table, dvipsnames]{beamer}
%\usepackage[utf8]{inputenc}

\usepackage[english]{babel}
\usepackage{fontspec} % loads: fixltx2e, etex, xunicode, fontspec.
\setmainfont{Arial}

\setlength{\parskip}{1ex}

\usepackage{fixme}

\usepackage{graphicx}
%\DeclareGraphicsExtensions{.pdf,.png,.jpg}
\graphicspath{{img/}}


%%%%%% Overpic + Textpos
\usepackage[permil]{overpic}

\usepackage[absolute,overlay,showboxes]{textpos}
\setlength\unitlength{1mm}
\setlength{\TPHorizModule}{1mm}
\setlength{\TPVertModule}{\TPHorizModule}
\textblockorigin{0mm}{0mm} % start everything near the top-left corner
\TPshowboxesfalse

%%%%%%%%%
\usepackage{color}

\usepackage{listings}
\definecolor{javared}{rgb}{0.6,0,0} % for strings
\definecolor{javagreen}{rgb}{0.25,0.5,0.35} % comments
\definecolor{javapurple}{rgb}{0.5,0,0.35} % keywords
\definecolor{javadocblue}{rgb}{0.25,0.35,0.75} % javadoc

\lstset{ %
    language=C++,                % choose the language of the code
    basicstyle=\tiny\ttfamily,       % the size of the fonts that are used for the code
    %numbers=left,                   % where to put the line-numbers
    %numberstyle=\footnotesize,      % the size of the fonts that are used for the line-numbers
    %stepnumber=1,                   % the step between two line-numbers. If it is 1 each line will be numbered
    %numbersep=5pt,                  % how far the line-numbers are from the code
    backgroundcolor=\color{white},  % choose the background color. You must add \usepackage{color}
    showspaces=false,               % show spaces adding particular underscores
    showstringspaces=false,         % underline spaces within strings
    showtabs=false,                 % show tabs within strings adding particular underscores
    frame=single,           % adds a frame around the code
    tabsize=2,          % sets default tabsize to 2 spaces
    %captionpos=b,           % sets the caption-position to bottom
    %breaklines=true,        % sets automatic line breaking
    %breakatwhitespace=false,    % sets if automatic breaks should only happen at whitespace
    %escapeinside={\%*}{*)}          % if you want to add a comment within your code
    keywordstyle=\color{javapurple}\bfseries,
    stringstyle=\color{javared},
    commentstyle=\color{javagreen},
    morecomment=[s][\color{javadocblue}]{/**}{*/},
}

\usepackage{multicol}

%%%%% Beamer %%%%%
%\usetheme{UniversiteitGent}
\usetheme{Madrid}
\usecolortheme{seagull}
\usepackage[german=guillemets]{csquotes}

\newlength{\templenght}
\newcommand{\textttb}[1]{\texttt{\bf#1}}

\usepackage{etoolbox}
\makeatletter
\patchcmd{\beamer@sectionintoc}{\vskip1.5em}{\vskip0.5em}{}{}
\makeatother

\AtBeginSection % Do nothing for \section*
{
	\begin{frame}[squeeze]
	\frametitle{Outline}
	\small
	\tableofcontents[currentsection]
\end{frame}
}

\newcommand{\red}[1]{{\color{red}#1}}
\newcommand{\green}[1]{{\color{ForestGreen}#1}}

\newcommand{\imagepos}[4]{
        \begin{textblock}{#1}(#2 , #3)
                \includegraphics[width=\textwidth]{#4}  
        \end{textblock}
}

\newcommand{\imageincl}[2]{
        \includegraphics[width=#1\textwidth,height=#1\textheight,keepaspectratio=true ]{#2}     
}

\newcommand{\imagebox}[2]{
        \includegraphics[width=#1, height=#1,keepaspectratio=true]{#2}  
}



\begin{document}
\author[Lechner]{Martin Lechner} %Teiniker/Schindler
\title{ITCS Coderetreat }
\subtitle{Graz 2019}
%\logo{}
%
%\subject{}
%\setbeamercovered{transparent}
\setbeamertemplate{navigation symbols}{}
\frame[plain]{\maketitle}

%%%%%%%%%%%%%%%%%%%%%%%%%%%%%%%%%% Frame %%%%%%%%%%%%%%%%%%%%%%%%%%%%%%%%%%%%%%%%%%
\begin{frame}{Welcome}

\imageincl{0.3}{ITCS_LOGO_small.jpg}

\centering
\Huge Welcome \\to the ITCS Code retreat\\ 2019!


%\imagepos{3}{1}{1}{ITCS_LOGO_small.jpg}
\end{frame}

%%%%%%%%%%%%%%%%%%%%%%%% Frame %%%%%%%%%%%%%%%%%%%%%%%%%%
\begin{frame}[squeeze]{Outline}
\small
\tableofcontents
\end{frame}

%CodeRetreat2019.png  ITCS_LOGO_small.jpg


%> 08:30 Eintreffen und Frühstück
%> 09:00 Begrüßung, Erklärung des Ablaufs, drei Aufgaben am Vormittag
%> 12:45 Mittagspause - für das Mittagessen ist gesorgt!
%> 13:45 Zwei Aufgaben am Nachmittag
%> 15:45 Wrap Up :)



\section{Introduction}

%%%%%%%%%%%%%%%%%%%%%%%%%%%%%%%%%% Frame %%%%%%%%%%%%%%%%%%%%%%%%%%%%%%%%%%%%%%%%%%
\begin{frame}{About me}
Martin Lechner (Ph.D. Informatics)
\pause
	\begin{itemize}[<+->]
		\item Software Developer/Scrum Master/Product Owner
		\item Work @ Unycom
		\item Teach at FH-Joanneum Kapfenberg
		\item Agilist and Clean Code advocate since 20 years		
	\end{itemize}  
\end{frame}

\subsection{What is a Code Retreat?}
%%%%%%%%%%%%%%%%%%%%%%%%%%%%%%%%%% Frame %%%%%%%%%%%%%%%%%%%%%%%%%%%%%%%%%%%%%%%%%%
\begin{frame}{\subsecname{}}
   A day where programmers come together to:
   \pause
	\begin{itemize}[<+->]
		\item hone their skills
		\item learn together and from each other
		\item focus on improving without pressure
		\item \textbf{\red{have fun!}}
	\end{itemize}  
\end{frame}


\begin{frame}{History of the code retreat}
	\begin{itemize}[<+->]
		\item Idea 2009 at Codemash Conference 
			\begin{itemize}[<*>]
				\item Question: How can we practice programming?
			\end{itemize}  
		\item Since 2009 practiced worldwide
			\begin{itemize}[<*>]
				\item Progagated by Corey Haines
				\item Global day of coderetreat 2018: 133 events worldwide
			\end{itemize}  
	\end{itemize}  
\end{frame}



\subsection{Structure, Problems and Rules}
\begin{frame}{Structure of our code retreat}
	\begin{itemize}[<+->]
		\item 5 Sessions: 45 min coding, 15 min retro + break
			\begin{itemize}[<+->]
				\item 3 Morning sessions
				\item appr. 12:45-13:45 Lunchbreak 
				\item 2 Afternoon sessions
				\item Wrap up
			\end{itemize}  
		\item Problem: Conway's Game of Life
			\begin{itemize}[<+->]
				\item Easy to understand - allows focus on code
				\item Complex enough for different approaches
				\item I expect this: \url{https://www.youtube.com/watch?v=-FaqC4h5Ftg}
				\item Just kidding - \textbf{purposely too big to finish}! :)
			\end{itemize}  
	\end{itemize}  
\end{frame}

\begin{frame}{Conway's Game of Life}
\pause
	\begin{itemize}[<+->]
		\item Infinite, two dimensional grid of cells
		\item Cell interacts with 8 neighbours
		\item At each step in time, a transitions occur
		\item Four rules for cells depending on neighbours
	\end{itemize}  
\end{frame}

\begin{frame}{Conway's Game of Life rules}
\pause
	\begin{enumerate}[<+->]
	  \item Any live cell with \textbf{fewer than two} live neighbours \textbf{dies}, as if caused by underpopulation.
      \item Any live cell with \textbf{two or three} live neighbours \textbf{lives} on to the next generation.		
	  \item Any live cell with \textbf{more than three} live neighbours \textbf{dies}, as if by overpopulation.		
	  \item Any dead cell with \textbf{exactly three} live neighbours \textbf{becomes a live cell}, as if by reproduction.
	\end{enumerate}
	\footnotetext{\url{en.wikipedia.org/wiki/Conway's_Game_of_Life}}
\end{frame}


\begin{frame}{Coding Dojo mindset}
A coding Dojo is a place to practice
\pause

	\begin{itemize}[<+->]
		\item Safe place outside work
		\item No pressure - slow down
		\item Experiment - Mistakes are Ok and necessary!
		\item Focus on doing it right (Perfectly) - not on getting it done
		\item Reduce gap between: \\"How we work" and "how we should work"!%\footnote[<+->]{Defines how much we suck ;)}
	\end{itemize}  
\end{frame}


\begin{frame}{Code retreat rules}
\pause
	\begin{itemize}[<+->]
		\item Pair Programming (PP)
		\begin{itemize}[<+->]
			\item Swap pairs each session
		    \item Try new partners, approaches, languages
		\end{itemize}		
		\item Test Driven Development (TDD)
		\begin{itemize}[<+->]
			\item Red - Green - Refactor
		    \item Babysteps
		\end{itemize}		
		\item After each session: Delete your Code!
		\item \textbf{\red{Really! Delete it!}}
		\begin{itemize}[<+->]
			\item Focus not on the output
		    \item Focus on the process: do it perfectly
		\end{itemize}		
		
	\end{itemize}
\end{frame}

\begin{frame}{Four rules of simple design (by Kent Beck)}
	Simple design:\pause
	\begin{itemize}[<+->]
		\item \textbf{\color{red} Passes the tests}
		\item Reveals all intention (Naming!)
		\item Contains no duplication (DRY)
		\item Fewest elements (KISS)
	\end{itemize}
\end{frame}


%%%%%%%%%%%%%%%%%%%%%%%%%%%%%%%%%% Frame %%%%%%%%%%%%%%%%%%%%%%%%%%%%%%%%%%%%%%%%%%
\begin{frame}[plain]
\begin{center}
\Huge
Any questions?

\pause
Lets Code!!!
\end{center}
\end{frame}



\section{Sessions}



\subsection{Session 1}
\begin{frame}[squeeze]{\subsecname{}}
	\textbf{\Large Get started!}
	\begin{itemize}
		\item Understand the problem 
		\item Get used to the setting (PP, TDD, Clean Code)
		\item No further constraints
	\end{itemize}
\end{frame}


\subsection{Session 2}
\begin{frame}[squeeze]{\subsecname{}}
	\textbf{\Large Pay attention to the working style}
	\begin{itemize}
		\item Fokus on PP, TDD, Clean Code
		\item Constraints: Ping-Pong Pairing
	\end{itemize}
\end{frame}

\subsection{Session 3}
\begin{frame}[squeeze]{\subsecname{}}
%Tdd as you mean it?
\end{frame}

\subsection{Lunchbreak}
\begin{frame}[squeeze]{\subsecname{}}
\end{frame}

\subsection{Session 4}
\begin{frame}[squeeze]{\subsecname{}}
% Mute ping pong
% Babysteps
\end{frame}

\subsection{Session 5}

%https://www.coderetreat.org/pages/hosting/hosts/

%Ping Pong
%Silent pairing
%Legacy code (Switch Laptop)
%5min Limit (Git)
%Tell dont ask
%No Primitives
%Classes as Verbs instead of nouns
%Immutables only, please (vielleicht mit etwas anderem kombinieren)
%No conditional statements
%If you feel people are not refactoring enough – choose one of the “Quality Constraints”. My favorite is – “No methods bigger than 5 lines”. Closing brackets do not count in those 5. For scripting languages like Ruby, Javascript and Python constraint is 3 lines.
%People are using a lot of primitives? “No Naked Primitives”
%To make them try new stuff or remember something forgotten try “No Loops” or “No If Statements”.

%https://docs.google.com/document/d/1W32Qh4zU2Y-QRdMkbF5zbqUGJM7wMmcM3DoukJLSLq8/edit

\section{Wrap up}
%%%%%%%%%%%%%%%% Frame %%%%%%%%%%%%%
\begin{frame}{\secname{} }
Closing Circle:
    \begin{itemize}[<*>]
		\item What did you learn today?
		\item What surprised you today?
		\item What will you do differently in the future?
		\item What will you change tomorrow? 
    \end{itemize}
\end{frame}

\section{Reference}
%%%%%%%%%%%%%%%%%%%%%%%%%%%
\begin{frame}{\secname{}}
    \begin{itemize}[<*>]
        \item Clean Code: A Handbook of Agile Software Craftsmanship
        \begin{itemize}[<*>] 
        \item Robert C. Martin
        \item 2008 Prentice Hall
        \end{itemize} 
     \end{itemize}
\end{frame}





\end{document}